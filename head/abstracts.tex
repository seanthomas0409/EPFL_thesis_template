%\begingroup
%\let\cleardoublepage\clearpage


% English abstract
\cleardoublepage
\chapter*{Abstract}
\markboth{Abstract}{Abstract}
\addcontentsline{toc}{chapter}{Abstract (English/Français)} % adds an entry to the table of contents
% put your text here
In the modern age of miniaturisation, Smart Materials, a type of material that reacts mechanically to a certain stimulus, have become an integral part of this revolution. Among these materials, Shape Memory Alloys (SMAs), who have the highest volumetric work density, are the ideal candidate in creating lightweight and miniature actuators. These alloys are a type of smart material that are able to revert back to their original shape when deformed as they are exposed to high temperatures. This exotic behaviour has allowed them to be the core active component in a plethora of applications such as grippers, bio-mimetic robots and surgical instruments.

Despite their high work density and their ability to function as artificial muscles, their implementation comes with some challenges. While the Shape Memory Effect (SME), the ability to recover strain when a thermal load is applied, is a remarkable behaviour, it is also a complex and multi-physical one. This complicates their design and makes it difficult to predict their behaviour. Furthermore, these alloys are only able to recover strain when deformed at low temperatures. This implies that a biasing element is required to exploit these materials in reversible actuators. Due to these limitations, the work density of SMAs, when implemented as actuators in robotic systems, are often much lower than their theoretical maximum.

In this thesis, the different types of SMA actuator implementations are studied from various applications to understand the design requirements and subsystems that are required to build an actuator. A holistic approach is, then, used to construct a design methodology to create highly integrated SMA-powered actuators in the hopes to prevent the degradation of the material's work density as is often the case in traditional SMA-based robotic systems. Here, in this work, the various identified subsystems of the SMA-powered actuator is combined to serve as a multi-functional element in the novel integrated SMA actuator.

In this thesis, different strategies are employed to integrate the SMA actuator into the robotic system while also proposing adapted sizing methodologies such that the resulting SMA-powered system can be sized to fit the required specifications to enable lightweight, compact and high bandwidth solutions. The work presents various case studies that utilise the proposed holistic design and sizing methodologies to serve as a proof-of-concept and to validate the methodology.

In this work, compliant and flexure-based mechanisms are exploited to exclude the need for a dedicated biasing element and create lightweight SMA-powered grippers to demonstrate the advantages of the proposed methodology. Additionally, utilising the mechanical behaviour of the SME, a lightweight mechanically-controlled crawling robot is designed and implemented. Furthermore, the work proposes, with the help of topology optimization and kirigami-inspired design, the creation of compliant SMA structures that allow the material to generate multiple outputs while remaining compact and easy to assemble. Lastly, a novel bistable gripper is designed and sized to experimentally validate the design methodologies proposed in this work.

The work demonstrates the different areas in which the degradation of the work density occur in traditional SMA systems. In this regard, design methodologies accompanied with sizing strategies are proposed that allows the creation of lightweight, high bandwidth and integrated SMA-based robotic systems. The results in this thesis, reveal the extraordinary value of SMAs in creating lightweight robotic systems and presents various strategies to allow the further integration of the alloy within the system.\\

\textbf{Keywords:} Shape Memory Alloys, Artificial Muscles, Compliant Mechanisms, Flexures, Mechanical-Intelligence, Buckled Beam, Topology Optimization, Kirigami, Bistable

% French abstract
\begin{otherlanguage}{french}
\cleardoublepage
\chapter*{Résumé}
\markboth{Résumé}{Résumé}
% put your text here
Les matériaux intelligents sont définissent un type de matériaux réagissant mécaniquement à certains stimulus. À l’ère moderne de la miniaturisation, ces derniers font désormais partie intégrante de cette révolution.

Les alliages à mémoire de forme (AMF) sont le candidat idéal pour créer des actionneurs légers et miniatures, car ils présentent la plus grande densité d’énergie au sein des matériaux intelligents. Après avoir été déformés, ces alliages peuvent reprendre leurs formes initiales lorsqu’ils sont soumis à des températures élevées. Ce comportement peu commun leur a permis de devenir le composant actif principal de nombreuses applications telles que des pinces, robots biomimétiques ou instruments chirurgicaux.
En dépit de leur densité d’énergie élevée et de leur capacité à fonctionner comme des muscles artificiels, leur utilisation s’accompagne de certains défis. Effectivement, même si l’effet à mémoire de forme (capacité à récupérer la déformation après chauffage), est un comportement remarquable, il par nature multi-physique et complexe à modéliser. Cela complique la conception de système les intégrants car il est ainsi plus difficile de prédire leur comportement. Par ailleurs, ces alliages ne sont capables de récupérer leur déformation que lorsqu’ils sont déformés à basse température. Ceci implique la nécessité d’un élément de polarisation pour utiliser ces matériaux dans des actionneurs réversibles. En raison de ces limitations, la densité d’énergie des AMF est souvent bien inférieure à leur maximum théorique lorsqu’ils sont implémentés comme actionneurs dans des systèmes robotiques.

% Afin de comprendre les exigences de conception, ainsi que les interactions entre les différents sous-systèmes, cette thèse propose d’étudier diverses applications d’actionneurs implémentant un AMF.
Une méthodologie de conception d’actionneurs hautement intégrés alimenté par l’AMF est développée de façon holistique afin de limiter au maximum la dégradation de densité d’énergie du matériau précédemment introduite. Cette méthodologie se base sur différentes combinaisons de sous-systèmes permettant l’obtention d’un élément multifonctionnel intégré dans l’actionneur AMF.

Afin d’obtenir des solutions légères, compactes et à large bande passante, cette thèse présente différentes stratégies d’intégration de l’actionneur AMF au sein du système robotique, tout en proposant des méthodologies de dimensionnement adaptées répondant aux spécifications requises.

Afin de valider la conception holistique proposée ainsi que les différentes méthodologies de dimensionnement, le travail est appliqué à diverses études de cas. Dans un premier temps, des mécanismes souples et flexibles sont exploités pour exclure le besoin d’un élément de polarisation dédié et créer des pinces légères alimentées par AMF. Ensuite, un robot rampant léger ne nécessitant pas de contrôle est développé et mis en œuvre en utilisant le comportement mécanique de l’AMF. De plus, des structures AMF compliantes sont proposées à l’aide de l’optimisation topologique et de la conception inspirée de Kirigami. Elles permettent au matériau de générer des sorties multiples tout en restant compact et facile à assembler.
Pour finir, un nouveau préhenseur bistable est conçu et dimensionné pour valider expérimentalement les méthodologies de conception proposées dans ce travail.\\

\textbf{Mots-clés:} Alliages à mémoire de forme, Muscles artificiels, Mécanismes compliant, Flexions, Intelligence mécanique, Lame flambée, Optimisation topologique, Kirigami, Bistable.
\end{otherlanguage}


% % German abstract
% \begin{otherlanguage}{german}
% \cleardoublepage
% \chapter*{Zusammenfassung}
% \markboth{Zusammenfassung}{Zusammenfassung}
% % put your text here
% \lipsum[1-2]
% \end{otherlanguage}

%\endgroup
%\vfill
