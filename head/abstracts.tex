%\begingroup
%\let\cleardoublepage\clearpage


% English abstract
\cleardoublepage
\chapter*{Abstract}
\markboth{Abstract}{Abstract}
\addcontentsline{toc}{chapter}{Abstract (English/Français)} % adds an entry to the table of contents
% put your text here
In the modern age of miniaturisation, Smart Materials, a type of material that reacts mechanically to a certain stimulus, have become an integral part of this revolution. Among these materials, Shape Memory Alloys (SMAs), who have the highest volumetric work density, are the ideal candidate in creating lightweight and miniature actuators. These alloys are a type of smart material that are able to revert back to their original shape when deformed as they are exposed to high temperatures. This exotic behaviour has allowed them to be the core active component in a plethora of applications such as grippers, bio-mimetic robots and surgical instruments.

Despite their high work density and their ability to function as artificial muscles, their implementation comes with some challenges. While the Shape Memory Effect (SME), the ability to recover strain when a thermal load is applied, is a remarkable behaviour, it is also a complex and multi-physical one. This complicates their design and makes it difficult to predict their behaviour. Furthermore, these alloys are only able to recover strain when deformed at low temperatures. This implies that a biasing element is required to exploit these materials in reversible actuators. Due to these limitations, the work density of SMAs, when implemented as actuators in robotic systems, are often much lower than their theoretical maximum.

In this thesis, the different types of SMA actuator implementations are studied from various applications to understand the design requirements and subsystems that are required to build an actuator. A holistic approach is, then, used to construct a design methodology to create highly integrated SMA-powered actuators in the hopes to prevent the degradation of the material's work density as is often the case in traditional SMA-based robotic systems. Here, in this work, the various identified subsystems of the SMA-powered actuator is combined to serve as a multi-functional element in the novel integrated SMA actuator.

In this thesis, different strategies are employed to integrate the SMA actuator into the robotic system while also proposing adapted sizing methodologies such that the resulting SMA-powered system can be sized to fit the required specifications to enable lightweight, compact and high bandwidth solutions. The work presents various case studies that utilise the proposed holistic design and sizing methodologies to serve as a proof-of-concept and to validate the methodology.

In this work, compliant and flexure-based mechanisms are exploited to exclude the need for a dedicated biasing element and create lightweight SMA-powered grippers to demonstrate the advantages of the proposed methodology. Additionally, utilising the mechanical behaviour of the SME, a lightweight mechanically-controlled crawling robot is designed and implemented. Furthermore, the work proposes, with the help of topology optimization and kirigami-inspired design, the creation of compliant SMA structures that allow the material to generate multiple outputs while remaining compact and easy to assemble. Lastly, a novel bistable gripper is designed and sized to experimentally validate the design methodologies proposed in this work.

The results in this thesis, reveal the extraordinary value of SMAs in creating lightweight robotic systems and presents various strategies to allow the further integration of the alloy within the system. The different areas in which the degradation of the work density are present in traditional SMA-powered systems and proposes design methodologies accompanied with sizing strategies to create lightweight, high bandwidth and integrated SMA-based robotic systems.\\

\textbf{Keywords:} Shape Memory Alloys, Artificial Muscles, Compliant Mechanisms, Flexures, Mechanical-Intelligence, Buckled Beam, Topology Optimization, Kirigami, Bistable

% French abstract
\begin{otherlanguage}{french}
\cleardoublepage
\chapter*{Résumé}
\markboth{Résumé}{Résumé}
% put your text here
À l'ère moderne de la miniaturisation, les matériaux intelligents, un type de matériau qui réagit mécaniquement à un certain stimulus, font désormais partie intégrante de cette révolution. Parmi ces matériaux, les alliages à mémoire de forme (AMF), qui présentent la plus forte densité de travail volumétrique, sont le candidat idéal pour créer des actionneurs légers et miniatures. Ces alliages sont un type de matériau intelligent capable de reprendre sa forme initiale lorsqu'il est déformé en étant exposé à des températures élevées. Ce comportement exotique leur a permis de devenir le composant actif principal d'une pléthore d'applications telles que les pinces, les robots bio-mimétiques et les instruments chirurgicaux.

Malgré leur densité de travail élevée et leur capacité à fonctionner comme des muscles artificiels, leur mise en œuvre s'accompagne de certains défis. Si l'effet de mémoire de forme (EMF), c'est-à-dire la capacité à récupérer la déformation lorsqu'une charge thermique est appliquée, est un comportement remarquable, il est également complexe et multi-physique. Cela complique leur conception et rend difficile la prédiction de leur comportement. En outre, ces alliages ne sont capables de récupérer leur déformation que lorsqu'ils sont déformés à basse température. Cela implique qu'un élément de polarisation est nécessaire pour exploiter ces matériaux dans des actionneurs réversibles. En raison de ces limitations, la densité de travail des SMA, lorsqu'ils sont implémentés comme actionneurs dans des systèmes robotiques, est souvent bien inférieure à leur maximum théorique.

Dans cette thèse, les différents types d'implémentations d'actionneurs SMA sont étudiés à partir de diverses applications afin de comprendre les exigences de conception et les sous-systèmes qui sont nécessaires pour construire un actionneur. Une approche holistique est ensuite utilisée pour construire une méthodologie de conception afin de créer des actionneurs hautement intégrés alimentés par le SMA dans l'espoir d'empêcher la dégradation de la densité de travail du matériau, comme c'est souvent le cas dans les systèmes robotiques traditionnels basés sur le SMA. Dans ce travail, les différents sous-systèmes identifiés de l'actionneur alimenté par le SMA sont combinés pour servir d'élément multifonctionnel dans le nouvel actionneur SMA intégré.

Dans cette thèse, différentes stratégies sont employées pour intégrer l'actionneur SMA dans le système robotique tout en proposant des méthodologies de dimensionnement adaptées afin que le système alimenté par le SMA puisse être dimensionné pour répondre aux spécifications requises pour permettre des solutions légères, compactes et à large bande passante. Le travail présente diverses études de cas qui utilisent la conception holistique proposée et les méthodologies de dimensionnement pour servir de preuve de concept et pour valider la méthodologie.

Dans ce travail, des mécanismes souples et flexibles sont exploités pour exclure le besoin d'un élément de polarisation dédié et créer des pinces légères alimentées par SMA pour démontrer les avantages de la méthodologie proposée. En outre, en utilisant le comportement mécanique du SMA, un robot rampant léger et mécaniquement contrôlé est conçu et mis en œuvre. En outre, le travail propose, à l'aide de l'optimisation de la topologie et de la conception inspirée de Kirigami, la création de structures SMA compliantes qui permettent au matériau de générer des sorties multiples tout en restant compact et facile à assembler. Enfin, un nouveau préhenseur bistable est conçu et dimensionné pour valider expérimentalement les méthodologies de conception proposées dans ce travail.

Les résultats de cette thèse révèlent l'extraordinaire valeur des SMA dans la création de systèmes robotiques légers et présentent diverses stratégies pour permettre l'intégration de l'alliage dans le système. Les différents domaines dans lesquels la dégradation de la densité de travail sont présents dans les systèmes traditionnels alimentés par des SMA et propose des méthodologies de conception accompagnées de stratégies de dimensionnement pour créer des systèmes robotiques légers, à large bande passante et intégrés à base de SMA.\\

\textbf{Mots-clés:} Alliages à mémoire de forme, Muscles artificiels, Mécanismes compliant, Flexions, Intelligence mécanique, Lame flambée, Optimisation topologique, Kirigami, Bistable.
\end{otherlanguage}


% % German abstract
% \begin{otherlanguage}{german}
% \cleardoublepage
% \chapter*{Zusammenfassung}
% \markboth{Zusammenfassung}{Zusammenfassung}
% % put your text here
% \lipsum[1-2]
% \end{otherlanguage}

%\endgroup
%\vfill
