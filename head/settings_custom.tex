% !TEX root = ../sethomas_thesis_main.tex
%%%%%%%%%%%%%%%%%%%%%%%%%%%%%%%%%%%%%%%%%%%%%%
%
%		Thesis Settings
%		Custom settings
%
%		2011
%
%%%%%%%%%%%%%%%%%%%%%%%%%%%%%%%%%%%%%%%%%%%%%%

% %
% %   Use this file for your own custom packages, command-definitions, etc...
% %
% %
% % Packages for references - cleverref must be last
% \usepackage{nameref}
% \usepackage{hyperref}
\usepackage{cleveref}
\usepackage{multicol}
\usepackage{multirow}
\usepackage{makecell}
\usepackage{overpic} % picture in picture
\usepackage{pict2e} % For the polygon function in overpic
\usepackage{siunitx}
\usepackage{tabu}
\usepackage{array}
\newcolumntype{P}[1]{>{\centering\arraybackslash}p{#1}} % New centered column with variable width
\usepackage{framed}
% \usepackage{tikz-dimline}
\usepackage{caption}
\usepackage{subcaption}
\usepackage{rotating}   % For sideways table
\usepackage{lscape}
\usepackage{afterpage}
\usepackage{standalone} % For seperature figure tex files

% \usepackage[shortlabels]{enumitem}
% % Reduce spacing in bibliography
% \setlength{\bibsep}{0pt plus 0.3ex}
% % Allow equations to break between pages
% \allowdisplaybreaks
% % Penalty for widow and orphan
% \widowpenalty=9999
% \clubpenalty=9999
% %Penalty for relation and binary operation breaks in equations
% \relpenalty=9999
% \binoppenalty=9999

%%%%%%%%%% FLOW CHART %%%%%%%%%%%%%
\usetikzlibrary{calc,trees,positioning,arrows,chains,shapes.geometric,%
    decorations.pathreplacing,decorations.pathmorphing,decorations.markings,shapes,%
    matrix,shapes.symbols,positioning,angles,quotes,patterns}

\tikzstyle{textbox} = [rectangle, rounded corners, minimum width=3cm, minimum height=1cm,text centered, text width=3cm, draw=black, fill=white]
\tikzstyle{arrow} = [thick,->,>=stealth]

%%%%%%%%%% CIRCUIT TIKZ %%%%%%%%%%%
\usetikzlibrary{spy,shapes.multipart,babel}
\usepackage[siunitx,europeanresistors,EFvoltages]{circuitikz} % To have the european convention for the drawing of electrical components. Also to have the current and voltage arrows going from high to low potential
\ctikzset{voltage/bump b=18pt,voltage/european label distance=15pt,voltage/distance from node=.1}

%%%%%%%%%% DIAGRAMS %%%%%%%%%%%%%
%See https://tex.stackexchange.com/a/29367/1952
\makeatletter
\tikzset{% customization of pattern
        hatch distance/.store in=\hatchdistance,
        hatch distance=5pt,
        hatch thickness/.store in=\hatchthickness,
        hatch thickness=5pt
        }
\pgfdeclarepatternformonly[\hatchdistance,\hatchthickness]{north east hatch}% name
    {\pgfqpoint{-1pt}{-1pt}}% below left
    {\pgfqpoint{\hatchdistance}{\hatchdistance}}% above right
    {\pgfpoint{\hatchdistance-1pt}{\hatchdistance-1pt}}%
    {
        \pgfsetcolor{\tikz@pattern@color}
        \pgfsetlinewidth{\hatchthickness}
        \pgfpathmoveto{\pgfqpoint{0pt}{0pt}}
        \pgfpathlineto{\pgfqpoint{\hatchdistance}{\hatchdistance}}
        \pgfusepath{stroke}
    }
\makeatother



%%%%%%%%%%%%%%%%%%%%%%%%%%%%%%%%%%%%%%%%%%%%%%
%
%       Custom Commands
%
%%%%%%%%%%%%%%%%%%%%%%%%%%%%%%%%%%%%%%%%%%%%%%
\newcommand{\todocite}{{\color{red}cite}}
% To circle some numbers (require the tikz package) :  \circled{<number>}
\newcommand*\bigcircled[1]{\tikz[baseline=(char.base)]{
            \node[shape=circle,draw,inner sep=2pt] (char) {#1};}}
\DeclareRobustCommand \circled[1]{\tikz[baseline=(char.base)]{
            \node[shape=circle,draw,inner sep=0.4pt] (char) {#1};}}


\newcommand{\degreeC}{$^{\circ}$C~}

% Thick lines for tabular
\makeatletter
\newcommand{\thickhline}{%
    \noalign {\ifnum 0=`}\fi \hrule height 1pt
    \futurelet \reserved@a \@xhline
}
\newcolumntype{"}{@{\hskip\tabcolsep\vrule width 1pt\hskip\tabcolsep}}
\makeatother

%%%%%%%%% FIGURE LEGENDS %%%%%%%%%%%%%

% Dotted line for legends
\newcommand{\dottedrule}{ \rule[2pt]{1pt}{0.5mm} \rule[2pt]{1pt}{0.5mm} \rule[2pt]{1pt}{0.5mm}}

\newcommand{\dhorline}[3][0]{%
    \tikz[baseline]{\path[decoration={markings,
      mark=between positions 0 and 1 step 2*#3
      with {\node[fill, circle, minimum width=#3, inner sep=0pt, anchor=south west] {};}},postaction={decorate}]  (0,#1+1.5pt) -- ++(#2,0);}}
\newcommand{\dvertline}[3][0]{%
    \tikz[baseline]{\path[decoration={markings,
      mark=between positions 0 and 1 step 2*#2
      with {\node[fill, circle, minimum width=#2, inner sep=0pt, anchor=south west] {};}},postaction={decorate}] (0, #1) -- ++(0,#3);}}

%%%%%%%%% ANNOTATING IMAGES %%%%%%%%%%%%%
\usepackage{tikz-imagelabels}

\imagelabelset{
   coarse grid color = red,
   fine grid color = gray,
   image label font = \sffamily\bfseries\small,
   image label distance = 2mm,
   image label back = black,
   image label text = white,
   coordinate label font = \sffamily\bfseries\small,
   coordinate label distance = 2mm,
   coordinate label back = none,
   coordinate label text = white,
   annotation font = \normalfont\small,
   arrow distance = 0mm,
   border thickness = 0.6pt,
   arrow thickness = 0.4pt,
   tip size = 1.2mm,
   outer dist = 0.1cm,
}
%%%%%%%%%%%%%%%%%%%%%%%%%%%%%%%%%%%%

% Inserts a scale bar into an image
% Optional argument 1: the colour of the bar and text
% Argument 2: an \includegraphics command
% Argument 3: the real world width of the image
% Argument 4: the length of the scale bar in pixels
% Argument 5: the length of the scale bar in mm
\newcommand{\scalebar}[5][white]{
 \begin{tikzpicture}
  \node[anchor=south west,inner sep=0] (image) { #2 };
  \begin{scope}[x={(image.south east)},y={(image.north west)}]
   \draw [#1, line width=0.4em] (0.04,1.2em) -- node[above,inner sep=0.3em, font=\normalsize] {\SI{#5}{\milli \meter}} (#5*#4/#3+0.04,1.2em);
  \end{scope}
 \end{tikzpicture}
}

%%%%%%%%%%%%%%%%%%%%%%%%%%%%%%%%%%%%%%%%%%%%%%%%%%%%%%%%%%%%%%%%%%%%%%
% LaTeX Overlay Generator - Annotated Figures v0.0.1
% Created with http://ff.cx/latex-overlay-generator/
% If this generator saves you time, consider donating 5,- EUR! :-)
%%%%%%%%%%%%%%%%%%%%%%%%%%%%%%%%%%%%%%%%%%%%%%%%%%%%%%%%%%%%%%%%%%%%%%
%\annotatedFigureBoxCustom{bottom-left}{top-right}{label}{label-position}{box-color}{label-color}{border-color}{text-color}
\newcommand*\annotatedFigureBoxCustom[8]{\draw[#5,ultra thick,rounded corners, opacity=0.8] (#1) rectangle (#2);\node at (#4) [fill=#6,ultra thick,shape=circle,draw=#7,inner sep=2pt,font=\sffamily,text=#8, opacity=0.8] {\textbf{#3}};}
%\annotatedFigureBox{bottom-left}{top-right}{label}{label-position}
\newcommand*\annotatedFigureBox[5]{\annotatedFigureBoxCustom{#1}{#2}{#3}{#4}{#5}{white}{#5}{#5}}
\newcommand*\annotatedFigureText[4]{\node[draw=none, anchor=south west, text=#2, inner sep=0, text width=#3\linewidth,font=\sffamily] at (#1){#4};}
\newenvironment {annotatedFigure}[1]{\centering\begin{tikzpicture}
\node[anchor=south west,inner sep=0] (image) at (0,0) { #1};\begin{scope}[x={(image.south east)},y={(image.north west)}]}{\end{scope}\end{tikzpicture}}

%%%%% ROUNDED CORNERS IMAGE %%%%%%
\newsavebox{\picbox}
\newcommand{\cutpic}[3]{
  \savebox{\picbox}{\includegraphics[width=#2]{#3}}
  \tikz\node [draw, rounded corners=#1, line width=4pt,
    color=white, minimum width=\wd\picbox,
    minimum height=\ht\picbox, path picture={
      \node at (path picture bounding box.center) {
        \usebox{\picbox}};
    }] {};}
%%%% GET WIDTH AND HEIGHT OF NODE %%%%%%%%
\makeatletter
% Stuff for calc compatiability.
\let\real=\pgfmath@calc@real
\let\minof=\pgfmath@calc@minof
\let\maxof=\pgfmath@calc@maxof
\let\ratio=\pgfmath@calc@ratio
\let\widthof=\pgfmath@calc@widthof
\let\heightof=\pgfmath@calc@heightof
\let\depthof=\pgfmath@calc@depthof
\makeatother
%%%%%%%%%%%%%%%%%%%%%%%%%%%%%%%%%%%%%%%%%%%%%%%%%%%%%%%%%%%%%%%%%%%%%%

%%%%%%%%%%%%%% COLORS %%%%%%%%%%%%%%%%%%%%%%%%%%%%%%%
\definecolor{myblue}{HTML}{0072BD}
\definecolor{mygreen}{HTML}{258F1B}
\definecolor{myred}{HTML}{C4000C}
\definecolor{myyellow}{HTML}{F2C335}
\definecolor{myblack}{HTML}{282C34}
\definecolor{darkgreen}{HTML}{0D330A}
\definecolor{lightgreen}{HTML}{32C326}
\definecolor{darkblue}{HTML}{09299E}
\definecolor{lightblue}{HTML}{367DD9}
\definecolor{lightgrey}{HTML}{6A6A6A}
\definecolor{darkgrey}{HTML}{1E1E1E}

\definecolor{lightcolor}{HTML}{97C8DB}
\definecolor{middlecolor}{HTML}{4090BE}
\definecolor{darkcolor}{HTML}{24498B}

%%%%%%%%%%%%%%%%%%%%%%%%%%%%%%%%%%%%%%%%%%%%%%%%%%%%%%%%%%%%%%%%%%%%%%%%%%%%%%%%
%%%%%%%%%%%%%%%%%%%%%%% VARIABLES %%%%%%%%%%%%%%%%%%%%%%%%%%%%%%%%%%%%%%%%%%%%%%
%%%%%%%%%%%%%%%%%%%%%%%%%%%%%%%%%%%%%%%%%%%%%%%%%%%%%%%%%%%%%%%%%%%%%%%%%%%%%%%%
\newcommand{\StrainRetention}{\ensuremath{\alpha_\epsilon}} % Strain
\newcommand{\Domain}{\ensuremath{\Omega}} % Design domain
\newcommand{\ObjFct}{\gamma} % Objective function
\newcommand{\IniOpt}[1]{\ensuremath{\widetilde{#1}^*}} % Optimal value of #1 when optimized alone
\newcommand{\ObjFctDec}{\gamma} % Infinideciaml evaluation of the Objective function
\newcommand{\EquFct}{h} % Equality constraint function
\newcommand{\IneqFct}{g} % Inequality constraint function
% \newcommand{\Pot}{\ensuremath{u}} % Function depending on the design variable (Usually a potential found with FEA)
\newcommand{\MIS}{\text{MIS}}  % Material Interpolation Function
\newcommand{\ArtDen}{\ensuremath{\rho}} % Artificial density
\newcommand{\FiltArtDen}{\widetilde{\ArtDen}} % Filtered Artificial Densities
\newcommand{\NbEQ}{NEC} % Number of equality constraints
\newcommand{\NbINEQ}{NIC} % Number of inequality constraints

\newcommand{\Volume}{\text{Vol}} %Volume of the resulting topology
\newcommand{\PenFac}{n} % penalization factor in SIMP model
\newcommand{\FiltRadius}{\ensuremath{r_\text{min}}} % radius of the filter (r_min usually imply a minimal lenghtscale)


\newcommand{\NSubObj}{\text{NF}} % Number of sub-objectives
\newcommand{\WeightSubObj}{\ensuremath{\alpha}}

\newcommand{\Lagran}{L}
\newcommand{\LagranParam}{\ensuremath{\lambda}}


%%%%%%%%%%%%%%%%%%%%%%%%%%%%%%%%%%%%%%%%%%%%%%%%%%%%%%%%%%%%%%%%%%%%%%%%%%%%%%%%
%%%%%%%%%%%%%%%%%%%%%% Structural %%%%%%%%%%%%%%%%%%%%%%%%%%%%%%%%%%%%%%%%%%%%%%
%%%%%%%%%%%%%%%%%%%%%%%%%%%%%%%%%%%%%%%%%%%%%%%%%%%%%%%%%%%%%%%%%%%%%%%%%%%%%%%%
\newcommand{\YoungModulus}{\text{E}} % Young modulus
\newcommand{\PoissonRatio}{\ensuremath{\nu}} % Poisson's ratio of the material $\ArtDen_\text{min}$
\newcommand{\Stiffness}{\text{k}} % stiffness coeficient
\newcommand{\Load}{\text{F}} % Load
\newcommand{\Disp}{\text{U}} % Displacements
\newcommand{\EStrain}{\text{S}} % Strain energy



%%%%%%%%%%%%%%%%%%%%%%%%%%%%%%%%%%%%%%%%%%%%%%%%%%%%%%%%%%%%%%%%%%%%%%%%%%%%%%%%
%%%%%%%%%%%%%%%%% Finite Element Analysis %%%%%%%%%%%%%%%%%%%%%%%%%%%%%%%%%%%%%%
%%%%%%%%%%%%%%%%%%%%%%%%%%%%%%%%%%%%%%%%%%%%%%%%%%%%%%%%%%%%%%%%%%%%%%%%%%%%%%%%
\newcommand{\ShapeFct}{\ensuremath{\phi}}
\newcommand{\StiffMat}{\text{K}}
\newcommand{\NDesElem}{\text{NE}}
\newcommand{\NNodes}{\text{NN}} % Number of nodes of the design elements
\newcommand{\NEROI}{\ensuremath{\text{NE}_\text{ROI}}}
\newcommand{\NROI}{\ensuremath{\text{N}_\text{ROI}}}
\newcommand{\VolumeElem}{\ensuremath{V}}
\newcommand{\LengthElem}{\ensuremath{l}}
\newcommand{\elemROI}{r}
\newcommand{\elem}{e}
\newcommand{\node}{n}
\newcommand{\elemJ}{j}


%%%%%%%%%%%%%%%%%%%%%%%%%%%%%%%%%%%%%%%%%%%%%%%%%%%%%%%%%%%%%%%%%%%%%%%%%%%%%%%%
%%%%%%%%%%%%%%%%%%%%%%%%%%%%%%%% Geometry %%%%%%%%%%%%%%%%%%%%%%%%%%%%%%%%%%%%%%
%%%%%%%%%%%%%%%%%%%%%%%%%%%%%%%%%%%%%%%%%%%%%%%%%%%%%%%%%%%%%%%%%%%%%%%%%%%%%%%%

\newcommand{\thickness}{t} % Thickness of the considered plane
\newcommand{\width}{w} % Width of the design domain
\newcommand{\height}{h} % Height of the design domain

%%%%%%%%%%%%%%%%%%%%%%%%%%%%%%%%%%%%%%%%%%%%%%%%%%%%%%%%%%%%%%%%%%%%%%%%%%%%%%%%%%
%%%%%%%%%%%%%%%%%%%%%%%%%%%% Induction heating %%%%%%%%%%%%%%%%%%%%%%%%%%%%%%%%%%%
%%%%%%%%%%%%%%%%%%%%%%%%%%%%%%%%%%%%%%%%%%%%%%%%%%%%%%%%%%%%%%%%%%%%%%%%%%%%%%%%%%

\newcommand{\VoltageEMF}{\ensuremath{V_\text{emf}}} % Back emf as a voltage drop
\newcommand{\TotalFlux}{\ensuremath{\Psi}} % Total flux seen by SMA
\newcommand{\frequency}{\ensuremath{f}} % Primary side frequency
\newcommand{\Inductance}{\ensuremath{L}} % Inductance of the flux linkage between SMA and primary coil
\newcommand{\ResisSMA}{\ensuremath{R_\text{SMA}}} %Effective resistance of the SMA
\newcommand{\InducedCurrent}{\ensuremath{I_\text{induced}}} %Indued current
\newcommand{\PJoules}{\ensuremath{P_\text{Joules}}} % Joules losses
\newcommand{\PrimaryCurrent}{\ensuremath{I_\text{prim}}} % Primary current
\newcommand{\AreaCoil}{\ensuremath{A}} % Effective area covered by the coil
\newcommand{\Nturns}{\ensuremath{N}} % number of turns

%%%%%%%%%%%%%%%%%%%%%%%%%%%%%%%%%%%%%%%%%%%%%%%%%%%%%%%%%%%%%%%%%%%%%%%%%%%%%%%%%%
%%%%%%%%%%%%%%%%%%%%%%%%%%%%%%%% Miscellaneous %%%%%%%%%%%%%%%%%%%%%%%%%%%%%%%%%%%
%%%%%%%%%%%%%%%%%%%%%%%%%%%%%%%%%%%%%%%%%%%%%%%%%%%%%%%%%%%%%%%%%%%%%%%%%%%%%%%%%%
\newcommand{\Xsymbol}{\mathbin{\protect\rotatebox[origin=c]{45}{$+$}}} % Used to define the X type of mandrel

\newcommand{\freqCritMONO}{\ensuremath{\frequency_c^\text{mono}}} % critical frequency for monolithic wire
\newcommand{\freqCritLITZ}{\ensuremath{\frequency_c^\text{litz}}} % critical frequency for Litz wire
