% !TEX root = ../sethomas_thesis_main.tex

\chapter{Overview of SMA Actuator Design}\label{chap:sma-actuator-design}
\section{Introduction}
In the scope of miniaturisation, creating highly integrated robotic system has become possible with the help of smart materials, notably Shape Memory Alloys (SMA). These alloys, having the highest work density, has made it possible to create miniature artificial muscles that can be integrated into compact and lightweight applications. SMAs have an interesting behaviour which consists of recovery any strain imposed on it when heated above a certain critical temperature threshold, often referred to as the Shape Memory Effect (SME). These alloys exploit the SME to create reversible actuators that are lightweight and compact. This effect is highly non-linear and is dependant on multiple variables, resulting in a highly complex and difficult to model behaviour. Thus, designing and sizing these alloys to create optimised actuators for complex applications is difficult and cumbersome.

In this chapter, an overview of the different implementations of SMA actuators are explored. In the context of different applications, the SMA actuators that are embedded in the different robotic systems are investigated. The traditional design methodology for these actuators are studied and presented in this chapter. The different examples of SMA actuators are used to create a conventional design methodology and the different subsystems of the robotic systems are studied. In this chapter, an in-depth look into the advantages and limitation of the methodology is conducted. In this manner, the conventional design methodology can be adapted into taking a holistic view of the robotic system and create a novel design approach that further promotes the integration of the SMA actuator subsystems into the final robotic system.
% application of sma actuators

\section{Basic Working Principle of the SME}
Shape Memory Alloys are subclass of smart materials that react to heat. This special brand of material where the can mechanically react with the help of some micro-structural changes when subjected to an external non-mechanical stimulus, in this case, temperature. The shape memory effect that occurs in this alloys occurs due to some phase transformation that happens when heated and cooled around a certain transition temperature. At low temperatures, the material exists in its Martensitic (M) phase where the material can be deformed easily. These deformations, similar to plastic deformation, results in the material being permanently deformed at these low temperatures. As the alloy is heated up its transition temperature, the material transforms from the M phase to the Austenitic (A) phase, recovering any of the \textit{permanent} strain imposed on it at low temperatures. This capacity to recover any strain imposed on it and return back to its original shape is often referred to as the shape memory effect. As the material cools below the transition temperature, the alloy returns back to the M phase, allowing the material to be \textit{plastically} deformed once again.
\section{Design Requirements of SMA Actuators}
\section{Traditional SMA Actuator Design}
\subsection{Types of SMA Actuators}
\subsection{Building Blocks of SMA Actuator Design}
\section{Summary and Conclusion}
