% !TEX root = ../sethomas_thesis_main.tex

\chapter*{Conclusion and Contributions}
\markboth{Conclusion and Contributions}{Conclusion and Contributions}
\addcontentsline{toc}{chapter}{Conclusion and Contributions}

% Introduce the goal and research objectives of the thesis
In this thesis, the goal of the work was to develop and demonstrate an novel approach to designing highly integrate and energy dense SMA-powered actuators. Initially, the research objective focused on the traditional methodologies and models when designing and sizing SMA actuators. Furthermore, the common types of implementations of such actuator were considered and their sizing methodologies were studied to deduce the various areas in which the work densities could be improved. This objective led to the development of a systematic design strategy that allowed creating integrated SMA-based actuators that improved the overall work density by transforming traditionally discrete subsystems into integrated multifunctional ones. Additionally, the traditional analytical and sizing models were adjusted to create an adapted sizing methodology so as to create actuators for any application.

% Introduce various prototypes
In this work, multiple case studies were presented that were designed and fabricated based on the proposed integrated SMA actuator design methodology. These include :

\begin{itemize}
    \item A multi-output lightweight gripper ideal for drone-based applications. This system consists of a compliant kinematic stage that also behaves as a biasing element for the SMA actuator and displays a high force-weight density.
    \item A highly lightweight untethered mechanically-controlled SMA inchworm robot that is able to produce a constant gait without the use of any sensors or control electronics.
    \item A simple kirigami-inspired linear actuator that behaves as a proof-of-concept for a high force-high stroke output actuator. This actuator manages to remain lightweight, compact and retain its high energy work density.
    \item A bistable gripper powered by a pair of antagonistic SMA helical springs capable of providing a gripping force without the need for any additional energy.
\end{itemize}

\section*{Contribution Statement}

This thesis aimed to contribute and explore the answers to the following key research questions:

\paragraph{How can the work density degradation of SMAs be prevented when designing actuators?}
In this work, various SMA-based actuators were studied from the major domains in which SMAs are employed. This study showed the key design considerations and criteria used in the design of SMA-powered actuators, the major one being the work density. This study also showed that in most cases, the traditional approach consists of creating an actuator where the key subsystems are discrete and lack integration. This results in the degradation of the work density as passive materials are added to the actuator without adding to the work output. This works answers the research question by proposing a holistic design methodology where the key subsystems present within an SMA actuator are integrated where the novel subsystems become multifunctional. In this thesis, a design methodology is presented to create SMA-based actuators that employ this holistic approach. This integration strategy has been employed on the various key elements of the SMA actuator such as the biasing element, kinematic and control stage. In each case, a case study is presented that makes use of this novel approach in the hopes to validate the design and sizing methodology. Furthermore, the key findings of this research question has led to numerous scientific publications and has contributed to the field by presenting novel techniques in designing SMA-powered actuators.

\paragraph{How can the traditional sizing methodology be adapted for integrated SMA actuators?}
This work demonstrates that by taking a holistic approach to creating SMA actuators can be beneficial when considering the work density of the resulting system. However, due to the multi-physical nature of the SME and the complex kinematic stages required, sizing such systems can be difficult. This works presents the traditional sizing methodology employed in SMA actuator designs and adapts the models such that they can be implemented in the proposed integrated SMA actuators. This sizing methodology allows the actuator to be designed in such a way that the least amount of the active material is required to drive the system. In most cases and in the case of the actuators presented in this work, the active SMA material is cooled passively using heat transfer with the surrounding air. Thus, the bandwidth of the resulting actuator depends directly on the cooling time required to return the SMA material back to the M phase. Thus, the sizing methodology presented and validated using various case studies, in this thesis, has allowed the actuator to be sized so as to maximize the bandwidth. Furthermore, the various adapted sizing methodologies have been described in the context of different SMA-powered actuators and published in various scientific publications.

\paragraph{Can a generative design approach be used to create and validate compliant SMA structures?}
The aforementioned integrated subsystem approach, in this work, was extended to other areas of the SMA actuator. Creating or generating compliant SMA actuators is proposed as a way to combine the functionality of the active material and the kinematic stage into one subsystem so as to further improve the work density of the final system. However, due to the complex nature of the SME creating complex geometries and predicting the resulting behaviour once heated can be difficult and computational expensive. Often, when a large design space is present with multiple variables, topology optimization can be used to algorithmically generate complex structures. In this work, topology optimization was presented and adapted to produce multi-output complaint SMA structures that can be be algorithmically generated with a relatively low computational cost. Furthermore, this work describes and validates a qualitative factor that can be used to test the efficiency of the generated design within the context of an SMA actuator. This simple algorithm was validated experimentally and has been published in a research publication so as to allow the creation of future 3D printed multi-output SMA actuators.

\section*{Future Implications}
This work has established a methodology that allows future engineers to create highly integrated and work dense SMA-powered actuators. With the advent of 3D printing technology and additive manufacturing of Ni-Ti based alloys, complex SMA structures can be fabricated and have already begun to show up on the forefront of research. Using the methodology and generative algorithms presented in this work, future researchers can exploit the approach to create multi-output and compact SMA-powered actuators capable of performing complex tasks.

Furthermore, the use of compliant structures paired with SMAs has been shown in this work to allow the creation of complex, yet, lightweight actuators. In future work, this design approach can be used to design and size lightweight actuators for various applications based on the required project specifications. In this work, the identified subsystems present in a traditional SMA-based actuator have been integrated so as create multi-functional elements and thus, reduce the overall work density degradation. This concept can be however extended to further incorporate the actuator into the entire robotic system for a greater holistic approach. In the case of a drone-ready gripper, the entire actuator can be integrated within the frame of the drone to further improve the work density of the system and render the robotic system even more lightweight or compact. Additionally, active cooling as a relevant subsystem in the actuator design has not been taken into account and can be another avenue of integration that can be considered to further extend this work while also increasing the bandwidth of the resulting actuator.

Lastly, in this work, the control strategy used for the proposed SMA-powered systems were simple without the use of any sensors. Furthermore, a mechanical control strategy was proposed that removed the need for control electronics or sensor further increasing the work density of the system. However, in future work, the sensorless control of the SMA element could be taken into account to allow the more complex SMA structures to also benefit from the removed sensors within the system, thereby, preventing work density degradation. Further developments in SMA manufacturing technologies can be harnessed with the use of the proposed design methodologies to create highly integrated, compact and lightweight actuators.
