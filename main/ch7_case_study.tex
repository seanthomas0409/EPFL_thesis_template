% !TEX root = ../sethomas_thesis_main.tex

\chapter{Validation using Novel SMA Gripper Systems}\label{chap:case-study}
\section{Introduction}
In this work, a design methodology is presented to design and size novel SMA actuators. By adapting the sizing methodology and design parameters of traditional SMA actuators, an approach is formed that can be exploited to create more compact and integrated solutions for actuators powered by shape memory alloys. In this final chapter, the novel design methodology is validated using different case studies. These case studies employ the strategies and analytical models presented in the previous chapters to adequately size compact and lightweight grippers. While validating the different analytical models, these case studies also validate the basic premise of the methodology.

The primary advantage of SMAs being their relatively high energy work density when compared to smart materials, they can be exploited to create powerful actuators for applications where low weight and compactness are required. In this context, a common application for SMAs are in actuators used to power grippers for drones or pick and place machine as shown in the work by \todocite and \todocite. Here, primary criteria to take into account are the overall dimensions, weight and time responses of the gripper. Using the design methodology presented earlier, the force and stroke requirements of the application can be taken in account such that the sizing of the SMA actuator can result in a gripper that is still integrated and fast.
\section{Case Study: A Multi-Output SMA Mandrel}\label{sec:smacm-mandrel}
\section{Case Study: A Bistable SMA Gripper}\label{sec:smabb-gripper}
\section{Conclusion}
