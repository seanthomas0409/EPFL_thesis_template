% !TEX root = ../sethomas_thesis_main.tex

\cleardoublepage
\chapter*{Introduction}
\markboth{Introduction}{Introduction}
\addcontentsline{toc}{chapter}{Introduction}
% Introduction to miniaturisation and Actuators
In the recent decades, there has been a wave of technological progress due to the advancements in miniaturisation. This trend to create smaller and more efficient devices has led to giant leaps in technological advancement. This era of miniaturisation, often referred to as the Second Industrial Revolution, was sparked by the creation of the integrated circuit. The exponential scaling and miniaturisation of silicon transistors has led to computers, filling entire rooms, being transformed into handheld devices that fit in one's pockets. This trend has allowed electronics to have faster performance, lower power consumption and be cheaper than its predecessor. This translates to market share as well as miniaturisation has given a competitive edge to technological and commercial products over its competitors. This age of miniaturisation has not been restricted to electronics but has also impact mechanical and optical devices. Motors, sensors and other such devices have all gone through the same trend in reduce footprint and increase performances resulting in a market where miniature actuators and sensors are readily available for relatively low prices.

% Introduction to Smart materials
The basic component of any robotic system is the actuator which is responsible for moving and controlling the system. The principle behind creating miniature robotics systems, thus, consists of downsizing actuators. The design and sizing criteria for actuators can not be necessarily applied from the macro scale when downsizing. However, the primary agent for the miniaturization of actuators has been the proliferation of Smart Materials. These materials , often referred to as artificial muscles, are able to provide some form of work when exposed to a certain stimulus such as stress, an electric or magnetic field.

% Shape memory alloys and its high work density

% Examples of SMA applications and its fields

% Commonality of such actuators (common design parameters)

% Challenges of designing SMA actuators

% Design Requirements

% Goal of the thesis

% How the goal is achieved
Smart materials and smart actuators design
\section*{Design Considerations}

\section*{Challenges of SMA Actuator Design}

\section*{Thesis Statement}

\section*{Thesis outline and Contributions}
